\documentclass[
	a4paper
]{scrreprt}

%%% PACKAGES %%%

% add unicode support and use german as language
\usepackage[utf8]{inputenc}
\usepackage[ngerman]{babel}

% make pdf/a
\usepackage[a-2b]{pdfx}
\pdfinfo{
	/Author (Pascal Stalder)
	/Title (Smart-Office People Counting)
	/Keywords (Infrarot, Image Processing, Smart-Home)
}

% Use Helvetica as font
\usepackage[scaled]{helvet}
\renewcommand\familydefault{\sfdefault}
\usepackage[T1]{fontenc}

% Better tables
\usepackage{tabularx}

% Better enumerisation env
\usepackage{enumitem}

% Use graphics
\usepackage{graphicx}

% Have subfigures and captions
\usepackage{subcaption}

% Be able to include PDFs in the file
\usepackage{pdfpages}

% Have custom abstract heading
\usepackage{abstract}

% Need a list of equation
\usepackage{tocloft}
\usepackage{ragged2e}

% Better equation environment
\usepackage{amsmath}

% Symbols for most SI units
\usepackage{siunitx}

\usepackage{csquotes}

% Clickable Links to Websites and chapters
\usepackage{hyperref}

% Change page rotation
\usepackage{pdflscape}

% Symbols like checkmark
\usepackage{amssymb}
\usepackage{pifont}

% Symbols like degree
\usepackage{gensymb}

% Make images stay where they belong
\usepackage{float}

\usepackage[absolute]{textpos}

% Glossary, hyperref, babel, polyglossia, inputenc, fontenc must be loaded before this package if they are used
\usepackage[sort=def]{glossaries}
% Redefine the quote charachter as we are using ngerman
\GlsSetQuote{+}
% Define the usage of an acronym, Abbreviation (Abbr.), next usage: The Abbr. of ...
\setacronymstyle{long-short}

% Bibliography & citing
\usepackage[
	backend=biber,
	style=apa,
	bibstyle=apa,
	citestyle=apa,
	sortlocale=de_CH
	]{biblatex}
\addbibresource{BAT.bib}
\DeclareLanguageMapping{ngerman}{ngerman-apa}

%%% COMMAND REBINDINGS %%%
\newcommand{\tabitem}{~~\llap{\textbullet}~~}
\newcommand{\xmark}{\ding{55}}

% Define list of equations - Thanks to Charles Clayton: https://tex.stackexchange.com/a/354096
\newcommand{\listequationsname}{\huge{Formelverzeichnis}}
\newlistof{myequations}{equ}{\listequationsname}
\newcommand{\myequations}[1]{
	\addcontentsline{equ}{myequations}{\protect\numberline{\theequation}#1}
}
\setlength{\cftmyequationsnumwidth}{2.3em}
\setlength{\cftmyequationsindent}{1.5em}

% Usage {equation}{caption}{label}
% \indexequation{b = \frac{\pi}{\SI{180}{\degree}}\cdot\beta\cdot 6378.137}{Bogenlänge $b$ des Winkels $\beta$ mit Radius 6378.137m (Distanz zum Erdmittelpunkt am Äquator)}{Bogenlaenge}
\newcommand{\indexequation}[3]{
	\begin{align} \label{#3} \ensuremath{\boxed{#1}} \end{align}
	\myequations{#3} \centering \small \textit{#2} \normalsize \justify }

% Todolist - credit to https://tex.stackexchange.com/questions/247681/how-to-create-checkbox-todo-list
\newlist{todolist}{itemize}{1}
\setlist[todolist]{label=$\square$}

%%% PATH DEFINITIONS %%%
% Define the path were images are found
\graphicspath{{./img/}{./pdf/}}

%%% GLOSSARY ENTRIES %%%
\makeglossaries
\newacronym{CNN}{CNN}{Convolutional Neural Network}
\newacronym{k-NN}{k-NN}{k-Nearest Neighbors}
\newacronym{SVM}{SVM}{Support Vector Machine}
\newacronym{IDE}{IDE}{Integrated Development Environment}
\newacronym{yolo}{YOLO}{You Only Look Once}
\newacronym{R-CNN}{R-CNN}{Region Convolutional Neural Network}
\newglossaryentry{Erode}{
	name={Erode},
	description={Morphologische Operation zur Verdünnung von Pixelflächen}
}
\newglossaryentry{Dilate}{
	name={Dilate},
	description={Morphologische Operation zur Expansion von Pixelflächen}
}

\newglossaryentry{Window}{
	name={Window},
	plural={Windows},
	description={Ein Bildausschnitt mit festgelegter Grösse}
}
\newglossaryentry{Padding}{
	name={Padding},
	description={Erweiterung eines Bildes in alle Richtungen mit bestimmten Pixelwerten}
}
\newglossaryentry{Cluster}{name={Cluster},description={Eine Gruppe von Datenobjekten mit ähnlichen Eigenschaften}}

%%% DOCUMENT %%%

\begin{document}

\include{HSLU_Preamble}

\pagenumbering{Roman}

\renewcommand{\abstractname}{Abstract}
\begin{abstract}
	\noindent
	Die stetig steigenden Ansprüche an Komfort und Energieeffizienz von Gebäuden führen dazu, dass moderne Gebäude sich ständig weiterentwickeln und intelligenter werden. Damit eine Gebäudesteuerung intelligente und nützliche Entscheidungen treffen kann, benötigt sie Informationen. Eine Art von solchen Informationen ist der Personenfluss. Um diese Informationen dreht sich diese Arbeit. Es sollte ein System entwickelt werden, welches automatisch die Anzahl und Positionen von Personen in einem Raum mittels Low Resolution Infrarotkameras bestimmen kann. Dabei soll besonderes Augenmerk auf Herausforderungen, die durch die Verwendung von Low Resolution Infrarotkameras auftreten, gelegt werden. Insbesondere sollen die Grenzen eines solchen Systems eruiert und aufgezeigt werden.\\
	\\
	Zur Umsetzung dieses Projekts wurden verschiedene Algorithmen und Methoden analysiert. Daraus wurden zwei Algorithmen ausgewählt. Zum einen ein Convolutional Neural Network und zur Referenz einen Threshold basierten Algorithmus. Diese beiden Algorithmen wurden so implementiert, dass sie parallel getestet und die Ergebnisse verglichen werden können. Um das neuronale Netz zu trainieren, wurde früh eine Datensammlung aufgebaut, um schnell ausreichend Daten zur Verfügung zu haben. Die benötigten Trainingsdaten wurden gelabelt, aufbereitet und dem Modell zum Training übergeben. Zuletzt wurden mehrere Testszenarien durchgespielt und die Ergebnisse dieser Versuche evaluiert.\\
	\\
	Beide Algorithmen sind sehr zuverlässig darin Personen zu erkennen. Obwohl in den Tests die Grenzen der Infrarottechnik ausgelotet wurden, konnten sie beide 92\% aller Personen erkennen. Die Threshold-Methode hat bei der Unterscheidung zwischen Personen und anderen Wärmequellen schlecht abgeschlossen, nur 74\% der Treffer waren Personen. Das Neuronale Netz hingegen erreichte 93\%.
	
\end{abstract}

\tableofcontents

\clearpage
\pagenumbering{arabic}

\chapter{Einleitung}

In diesem Digitalen Zeitalter werden Ansprüche an Komfort und Energieeffizienz immer grösser. Damit auch moderne Gebäude diesen Ansprüchen gerecht werden müsse sie sich ständig weiterentwickeln und intelligenter werden.\\
Das ideale Gebäude hat stets die perfekte Temperatur, Luftfeuchtigkeit und Luftqualität, erreicht dies mit minimalem Energieaufwand und ohne dass der Nutzer dafür einen Finger rührt. Um dies zu erreichen benötigt die Gebäudesteuerung Informationen, mit denen sie die optimalen Bedingungen errechnen kann. Damit der Nutzer keinen Aufwand hat, müssen diese Informationen vollautomatisch gesammelt und verarbeitet werden.\\
Ein zusätzlicher Nutzen dieser Daten ist die erstellung von Auslastungsstatistiken. Damit kann beispielsweise in einer Firma die Ausnutzung von Räumlichkeiten ausgewertet werden und mithilfe dieses Wissens die Infrastruktur optimiert werden.

\section{Ausgangslage}
\label{sec:Ausgangslage}

Gebäudesteuerungen sollen in Zukunft mehr auf die Belegungen und das Verhalten der Nutzer eingehen, anstatt einem fixen Schema zu folgen. Voraussetzung dafür ist, dass die Steuerung, die Belegung und das Verhalten der Nutzer kennt.\\
Dafür soll in dieser Diplomarbeit das Potential kostengünstiger thermischer Kamerasystemen, zur Erfassung des Nutzerverhaltens in Büroräumen evaluiert werden. Mittels State-of-the-Art Bildanalyse und Deep-Learning soll versucht werden, die Belegung und den Personenfluss eines Sitzungszimmers zu bestimmen.\\
Ein Sitzungsraum der Hochschule Luzern in Horw wurde dafür mit zwei Infrarotkameras ausgerüstet. Zusätzlich steht eine herkömmliche Kamera als Referenz zur Verfügung.


\section{Zielsetzung}
\label{sec:Zielsetzung}

Ziel dieses Projekts ist es ein Modell zu erstellen, welches akkurat die Anzahl Personen und deren Position, in einem Raum, mittels einem Infrarotsensor, bestimmen kann. Dazu wird in einem ersten Schritt der Stand der Technik evaluiert und eine Methode für das weitere Vorgehen festgelegt. Danach wird diese Methode implementiert und unter realen Bedingungen getestet. Dabei soll Hauptsächlich gezeigt werden, ob ein solches System praktikabel ist und welche Herausforderungen oder Probleme es mit sich bringt. Zudem soll evaluiert werden wie diese Probleme überwunden werden können.

\section{Projektanforderungen}
\label{sec:Requirements}

Es sollen folgende Punkte erfüllt werden

\begin{itemize}
	\item Der Stand der Technik im Bereich Objekterkennung auf Infrarotbildern soll ermittelt werden.
	\item Die gefunden Algorithmen sollen evaluiert werden, nach ihrer Tauglichkeit für diese Problemstellung.
	\item Es soll ein lauffähiges System entwickelt werden, welche Position und Anzahl der Personen, auf einem Infrarotbild, ausgibt.\\
	Dazu sollen ein bis zwei Algorithmen ausgewählt und Implementiert werden.
	\item Es soll aufgezeigt werden, welche Schwierigkeiten bei der Entwicklung eines solchen Systems beachtet werden müssen und wie diese überwunden werden können.

\end{itemize}





\chapter{Stand der Technik}
\label{ch:StandDerTechnik}

Die Bildverarbeitung im Bereich von Infrarot- und Temperaturbildern baut auf den erforschten Methodiken der klassischen Bildverarbeitung auf. Dies zeigt auch der Bericht einer Forschungsgruppe der ETH-Zürich \parencite{gomez2018thermal}. Diese stützt sich bei der Erarbeitung ihres Algorithmus zur Infrarot-Bildverarbeitung auf «A Convolutional Neural Network Cascade for Face Detection» \parencite{li2015convolutional}. Aus diesem Grund befindet sich im folgenden Stand der Forschung sowohl Algorithmen der klassischen Bildverarbeitung als auch solche, die aus dem Feld der Verarbeitung von thermischen Aufnahmen stammen.\\
Es wurden einige Publikationen zu Objekterkennung im Infrarobereich gefunden, die auf die Verwendung von \gls{SVM} aufbauen \parencite{suard2006pedestrian, bertozzi2003pedestrian, zhang2007pedestrian}. Diese sind leider aufgrund ihres Alters problematisch. Alle diese Publikationen stammen von 2007 oder früher. Da aber um das Jahr 2011 bei der Mehrzahl von Wettbewerben \gls{SVM}'s von \gls{CNN}'s überholt wurden \parencite{Historyo5:online}, können diese nicht mehr vorbehaltlos als aktueller Stand der Technik in der Objekterkennung gewertet werden.

\section{Technologische Grundlagen}
\label{sec:technicalBase}
Die momentan gängigsten Algorithmen sind \gls{CNN} \parencite{li2015convolutional} und verschiedene Filtermethoden, wie zum Beispiel Hough-Transformationen \parencite{ye2015new}. Auch eine Kombination von \gls{CNN} mit vorangehenden Filtern sieht man oft. Zudem ist es gängige Praxis ein zu bearbeitendes Bild nicht als Ganzes zu verwenden, sondern jeweils nur einen kleinen Teil des Bildes zu verarbeiten. Dabei wird die Grösse dieses Fensters so gewählt, dass das grösstmögliche gesuchte Objekt vollständig abgebildet werden kann. Das Fenster wird in kleinen Schritten über das Bild bewegt, um jeden dieser Bereiche zu analysieren. Dies ermöglicht eine sehr grobe Vorsortierung, bei der alle Teile, die zum Beispiel nur uninteressanten Hintergrund enthalten, direkt verworfen werden können. Dabei wird gleichzeitig die Performance gesteigert und die Analyse der interessanten Teile vereinfacht, da diese nun konzentrierter sind und nicht von der Position im Bild abhängen.\\
\\
\gls{CNN}’s trainieren eine Vielzahl von Filtern, welche die gesuchten Features bestmöglich repräsentiert. Da dieses Produkt als Bild für Menschen jedoch keinen Sinn ergeben würde, wird es direkt mit einer Aktivierungsfunktion in eine Wahrscheinlichkeit oder eine Klassifizierung umgewandelt.\\
Die Kaskadierung von \gls{CNN}’s ist eine Methode, bei welcher schrittweise komplexere und rechenintensivere \gls{CNN}’s nacheinander verwendet werden. Dies hat den Vorteil, dass Bildausschnitte, die nur Hintergrund enthalten, sehr schnell durch simplere \gls{CNN}'s verworfen werden können. Entscheidet das simple \gls{CNN}, dass es sich nicht um Hintergrund handelt, werden komplexere \gls{CNN}’s verwendet, um trotzdem eine sehr hohe Präzision zu erhalten.\\
\\
Um mittels Hough-Transformationen Formen zu erkennen, wird zuerst mit einem Kantendetektionsalgorithmus, wie z.B. dem Canny-Algorithmus, ein Binärbild erstellt, welches alle Kanten des Originalbildes repräsentiert. Im nächsten Schritt werden dann mittels Hough-Transformationen Formen, wie Linien oder Kreise, in diesem Binärbild identifiziert. Aus diesen Formen kann danach abgeleitet werden, ob es sich um ein potenziell relevantes Objekt handelt oder nicht.\\
\\
Weiter gibt es auch noch Methoden wie \gls{k-NN} oder \gls{SVM}. Diese werden aber immer weniger verwendet, da sie in nahezu jedem Anwendungsfall von neuronalen Netzen überholt wurden. \gls{k-NN}'s werden oft als erste simple Kontrollversuche oder für einfache Problemstellungen, sowie in Kombination mit anderen Methoden, verwendet. \gls{SVM}'s waren lange die besten Kandidaten in Objekterkennungs-Wettbewerben, wie zum Beispiel ImageNet \parencite{ILSVRC15}. Seit die nötige Rechenleistung verfügbar ist, um tiefe \gls{CNN}'s zu trainieren, gewannen in solchen Challenges fast ausschliesslich \gls{CNN}'s. Momentan werden \gls{SVM}'s meist zur Unterstützung anderer Methoden eingesetzt.\\
\\
Im Bereich von \gls{CNN} gibt es viele Variationen wie \gls{yolo} oder der \gls{R-CNN} \parencite{yoloRCnn}, die sich auf das Detektieren von mehreren Objekten in einem Bild spezialisieren. Von diesen gibt es wiederum eine Vielzahl von Ableitungen die Geschwindigkeit oder Genauigkeit optimieren. Diese sind zwar meist auf hochauflösende Farbbilder spezialisiert, können aber konzeptuell auch in solch einem Projekt angewendet werden.



\chapter{Ideen und Konzepte}

Gemäss der Problemstellung sollten passende Algorithmen gesucht und Analysiert werden. Dazu wurde in der Ersten Projektphase eine breite Recherche durchgeführt wobei in erster Linie verschiedene Lösungsansätze gesucht wurden.

\subsection{Algorithmensuche}

Um Diese Aufgabe zu lösen wird ein bewährter und passender Algorithmus benötigt. Dazu wurde in der Recherchephase ermittelt welche Arten von Algorithmen für solche Fälle verwendet werden. Die gefundenen Algorithmen, werden dann evaluiert und bewertet so, dass schlussendlich ein Algorithmus ausgewählt werden kann der bestmöglich zur Lösung der Aufgabe verwendet werden kann.

\subsection{Clustering von Wärmebereichen}

Es werden Cluster von Pixeln über einem bestimmten Temperatur-Threshold gebildet. diese werden einzeln analysiert und anhand von Temperaturverteilung innerhalb des Clusters und der Form des Clusters wird entschieden ob es sich um eine Person handelt. Sind die Cluster evaluiert und markiert als Person oder keine Person können die Zentren der Person-Cluster direkt als Treffer verwendet werden.


\subsection{K-Nearest Neighbors}

\gls{k-NN}'s ist ein simpler Algorithmus das Datenpunkte vergleicht und danach entscheidet welche die grösste Ähnlichkeit besitzen. dieser Algorithmus wird ähnlich dem \gls{CNN} mit Referenzdaten bestückt und entscheidet daraufhin zu welcher Klasse ein Bild die grösste Ähnlichkeit aufweist. 

\subsection{Hough-Transformationen}

Hough-Transormationen erkennen Formen, wie Kreise und Linien auf Bildern. Dazu muss das Bild zuerst mittels ein Kantendetektionsalgorithmus so bearbeitet werden, dass nur noch kanten zu sehen sind. Danach extrahiert man mittel Hough-Transformationen Kreise und Linien. Diese Methodik könnte in diesem Projekt zur Unterstützung andere Algorithmen verwendet werden.


\subsection{Thresholding}

Beim Thresholding wird das Infrarotbild mittels mehreren fixierten Werten in ein Binärbild umgewandelt in welchem dann nur noch weisse Flecken zu sehen sein sollten, die Personen repräsentieren. Dazu wird in diesem Fall eine Mindest- und Maximaltemperatur festgelegt. Alle Pixel die innerhalb dieses Bereichs liegen, werden weiss eingefärbt, alle anderen schwarz. Danach wird dieses Bild erodiert und dilatiert. Bei der Erosion wird mit einer kleine Matrix, ein sogenannter Kernel, über das Binärbild iteriert. Immer, wenn mindestens ein Feld des Kernels schwarz ist wird die gesamte Fläche des Kernels, auf dem resultierenden Bild schwarz eingefärbt. die Dilatation funktioniert nach dem gleichen Prinzip aber arbeitet dabei mit den weissen Pixel.

\subsection{Convolutional Neural Network}

Während der Recherche stellte sich heraus, dass das \gls{CNN} für diesen Bereich der Objekterkennung, eine sehr beliebte Variante ist. Die Mehrheit der Arbeiten die zu diesem Thema gefunden wurden verwendeten \gls{CNN} in irgend einer Form.\\
\gls{CNN} funktionieren wie in Kapitel \ref{ch:StandDerTechnik} erklärt indem sie Filter trainieren welche aus dem ursprünglichen Bild die nötigen Informationen extrahieren um zu bestimmen zu welcher Klasse das Bild gehört. Um dieses Training zu ermöglichen werden sehr viele Trainingsdaten benötigt.

\section{Datensammlung}

Da einige der erwähnten Methoden einen relativ grossen Datensatz benötigen, um Trainiert zu werden, stellen sich die Fragen wie diese gesammelt und vor allem, wie diese gelabelt werden. Das Sammeln an sich kann durch ein simples Skript realisiert werden, indem die Infrarotbilder periodisch von den Infrarotkameras angefordert und abgespeichert werden. Zusätzlich wurden parallel dazu auch Bilder einer optischen Kamera im Raum persistiert.\\
Das Labeln der Infrarotbilder konnte auf zwei Arten umgesetzt werden. Man Labelt alle Bilder die verwendet werden sollen manuell mit Hilfe eines Labeling Tools. Oder man erstellt ein Komplexes Program, das mittels 'State of the Art' Bildverarbeitung aus den optischen Bilder die Personen erkennt und deren Position auf die Position im Infrarotbild umrechnet.\\
\\
Dieser zweite Ansatz wäre vollautomatisch, schnell und wiederverwendbar, jedoch wäre die Erstellung eines solchen Programms sehr aufwendig und es könnte auch nicht garantiert werden, dass alle Personen korrekt markiert wurden. Es müsste also trotzdem Manuell kontrolliert werden, ob das Labeln korrekt ablief. 



\chapter{Methode}

In diesem Kapitel wird beschreiben, wie in dieser Arbeit vorgegangen wurde und welche Technologien und Techniken eingesetzt wurden.


\section{Vorgehen}

Da dies ein Innovationsprojekt ist, wurde in Explorativen Problemlösungszyklen gearbeitet. Dies alles im Rahmen einer Grobplanung, die diverse Meilensteine festlegte die zu erreichen waren.\\
Das Projekt teilte sich in drei Phasen auf.  Die Intialrecherche, die Implementierung der ausgewählten Algorithmen und die Evaluation des Systems.\\
Während der Initialrecherche wurde der Stand der Technik, in der Objekterkennung im Infrarotbereich, ermittelt. Danach wurden die gefundenen Algorithmen evaluiert und daraus zwei Algorithmen ausgewählt. Da der zeitliche Rahmen dieses Projekts es nicht zuliess, bei der Auswahl der Algorithmen alle gefundenen Methoden ausgiebig zu testen, wurde vorallem darauf geachtet ob und wie diese in ähnlichen Projekten erfolgreich eingesetzt wurden.

\section{Technologien \& Frameworks}
Zur Umsetzung dieses Projekt wurde Python 3.7 mit Unterstützung durch die Frameworks Tensorflow \parencite{tensorflow2015}, Keras \parencite{keras} und OpenCV \parencite{opencv} verwendet. Zudem wurden weitere Open Source Python Packages verwendet, zum Beispiel Numpy \parencite{numpy} um die Implementation zu vereinfachen.\\
Als \gls{IDE} wurde Pycharm von Jetbrains \parencite{pycharm} verwendet. Die zur Verfügung gestellten Infrarotkameras sind HTPA80x64d von Heimann Sensor GMBH.

\section{Techniken}

Wie in Kapitel \ref{ch:ideasAndConcepts} erwähnt wurde wurden das \gls{CNN} und die Threshold-Methode ausgewählt um in diesem Projekt zur Lösung der Aufgabe verwendet zu werden. 

\subsection{CNN}

Es wird ein \gls{CNN} trainiert und dazu verwendet einzelne Bildausschnitte als 'Person' oder 'keine Person' zu klassifizieren. Dazu werden Infrarotbilder gelabelt, um diese Label ein 16x16 Fenster ausgeschnitten und dem \gls{CNN} zum Trainieren übergeben.\\
Um danach ein Infrarotbild auszuwerten und die Personen darauf zu bestimmen, wird mit einem Sliding Window über das Bild gefahren und jeder ausschnitt dem \gls{CNN} zu Klassifizierung übergeben.

\subsubsection{Windowing}
Um das Training des \gls{CNN} zu vereinfachen wird Windowing eingesetzt. Wodurch das \gls{CNN} nur noch Klassifizieren muss und nicht auch noch die Position der Person bestimmen.

\subsection{Threshold-Methode}

Bei der Threshold Methode werden mittels einigen Beispielbildern Parameterwerte festegelegt, mit denen bestmöglich die Personen aus einem Infrarotbild gefiltert werden können. Das Bild wird mittels diesen Parametern auf ein Binärbild reduziert, in dem idealerweise pro Person eine geschlossene weisse Fläche zu sehen ist. Die Zentren dieser Flächen entsprechen dann der Position der Personen.

\chapter{Realisierung}

\section{Systemübersicht}

\begin{figure}[H]
	\centering
	\includegraphics[width=.8\linewidth]{SystemOverview}
	\caption{Grobübersicht des Systems mit CNN- und Thresholdalgorithmus}
	\label{SystemOverview}
\end{figure}

\section{UDP-Schnittstelle}

Um die Bilder automatisiert erhalten zu können wurde eine Schnittstelle zu den Infrarotkameras implementiert. Es wurde das UDP-Protokoll verwendet, da die Kameras nur über dieses Protokoll angesprochen werden konnten.\\
 Die einzige ander Möglichkeit and Bilder zu gelangen wäre über die GUI Applikation des Herstellers Videos aufzunehmen. Dies sind bereits auf RGB convertiert, beihalten also nicht alle Information des ursprünglichen Bildes und die Applikation kann nicht automatisiert angesprochen werden. Folglich ist dies keine praktikable Variante um effizient Daten zu sammeln.\\
\\
Die Kameras können ausserdem nur mittels UDP-Broadcasts identifiziert und and einen Socket gebunden werden. Aus diesem Grund konnten die Kameras nicht im allgemeinen Schulnetzwerk installiert werden, sondern mussten in einem separaten Netzwerk betrieben werden. In diesem Netzwerk konnten sie über einen Computer angesteuert werden, der widerum mit dem HSLU-Netzwerk verbunden ist. Um die Entwicklung der Schnittstelle möglichst einfach zu gestalten wurde versucht einen Remoteinterpreter aufzusetzen. Dies ist eine funktion von Pycharm \parencite{pycharm} erlaubt es, auf einem Lokalen System zu entwickeln aber die Software direkt auf einem Remote System auszuführen und zu Debuggen.\\
Leider wird dies für Windows Remote Systeme nicht unterstützt. Deshalb wurde der Code jeweils manuell via sftp auf den Computer im Sitzungszimmer kopiert. Danach wurde mittel Windows Remotedesktopverbindung die Software ausgeführt und getestet. \\
\\
In einer ersten Variante wurde in dieser Schnittstelle mit Einzelbilder gearbeitet. Dies bietet den Vorteil, dass man die Bildrate einfah definieren und steuern kann. Leider war die Qualitiät dieser Bilder sehr schlecht. Durch eine Rücksprach mit dem Hersteller stellte sich heraus, das die Verwendung der Streaming Funktion der Kameras qualitativ besser Bilder liefert. Deshalb wurde die Schnittstelle auf die Verwendung von Streams abgeändert. Dabei war die Generator Funktionalität von Python sehr hilfreich. Die Eigenschaft von Generators, dass diese erst ausgeführt werden, wenn ein Objekt angefragt wird, konnte genutzt werden, um jederzeit das aktuellste Bild zu erhalten.

\section{Datensammlung}

Um die Trainingsdaten effizient zu sammeln wurde ein  Skript erstellt, welches von beiden Kameras gleichzeitig Bilder anfordert und Abspeichert. Dieses musste über mehrere Monate unterbruchlos laufen und wurde deshalb so Implementiert, dass es sich bei einem Fehler automatisch neu startet. Zusätzlich wurden parallel dazu auch Bilder der Referenzkamera aufgezeichnet, um das Labeling der Infrarotbilder zu vereinfachen und als Ground Truth zu verifizieren.\\
Alle aufgezeichneten Bilder wurden lokal auf einer Festplatte des Computers im Sitzungszimmer gespeichert. Um alle Bilder eindeutig zu identifizieren, wurden sie mit Art des Bildes, Infrarot oder Ground Truth, Zeitstempel und bei den Infrarotbildern mit Kamera 1 oder 2 versehen. Um das ganze übersichtlicher zu gestalten wurde das ganze in einem Ordnersystem abgelegt, das wie folgt aufgebaut ist.

\begin{itemize}
	\item ../ [Jahr] / [Monat] / [Tag] / [Bildtyp]\_[Uhrzeit]\_[Kamera].[Dateityp]
	\item Bsp.: ../2019/05/23/IR\_Image\_10\_33\_45\_2.npy
\end{itemize}


\section{CNN}

Um das \gls{CNN} Trainieren zu können müssen die Bilder gelabelt, gepaddet, Fenster extrahiert, und normalisiert werden. Dazu wurden verschiedene Module implementiert, die diese Aufgaben übernehmen.

\subsection{Labeln}
\label{sec:labeling}

Für das Labeln der Bilder wurde \textit{labelme}\parencite{labelme2016} verwendet da dies einfaches und praktisches User Interface bietet. Die Labels wurden jeweils in dem Verzeichnis \textit{labels} abgelegt, der auf der gleichen ebene wie das Verzeichnis \textit{ir\_images} in dem die dazugehörigen Infrarotbilder abgelegt sind.\\
Die rohen Infrarotbilder wurden als .npy \parencite{npyformat} Files gespeichert. Dies, weil die Pixel der Bilder in Kelvin*10 sind, d.h. 2731 entspricht 0\degree C. Aus diesem Grund konnten sie nicht auf einfache Art in ein Grafikformat persistiert werden. Zudem ist es wünschenswert Berechnungen mit den Originalwerten durchzuführen.\\
Um die Infrarotbilder aber labeln zu können wird ein Bild mit .jpg oder ähnlichem Format benötigt. Dazu wurde das Bild auf Graustufen reduziert. Dabei geht zwar ein Teil der Information verloren, aber man kann genug erkennen um die Bilder korrekt zu Labeln. zudem konnte im Zweifelsfall das RGB Bild der Referenzkamera zu Hilfe gezogen werden. Da die konvertierten Bilder die gleichen Dimensionen aufweisen können die definierten Labels direkt auf die Infrarotbilder angewendet werden.

\subsection{Preprocessing der Trainingsdaten}

Um das vorbereiten der Trainigsdaten möglichst einfach zu gestalten wurde die Klasse Loader implementiert. Dies bietet die Methode \textit{load\_data\_by\_labels()} welche für ein spezifiziertes Verzeichnis alle Objekte lädt für die ein Label existiert. Diese Methode basiert darauf das die Labels und Infrarotbildern wie vorgängig erwähnt im selben Verzeichnis und in den Ordnern labels und ir\_images abgelegt wurden. Die Methode kann zudem mit mehreren Parameter angepasst weden.

\subsubsection{Loader}

\noindent -Konstruktor Parameter:
\begin{itemize}[leftmargin=*, labelindent=3cm, labelsep=1cm]
	\item[\textit{source\_folder}] String: Das Verzeichnis aus dem die Trainingsdaten geladen werden sollen, muss die Ordner \textit{ir\_images} und \textit{labels} beinhalten.
	\item[\textit{window\_size}] (int, int): Die grösse des Fensters das um das Label extrahiert werden soll.
	\item[\textit{extend\_by\_roaming}] Boolean: Um jedes Label wird in einem Umkreis drei Pixeln zusätzliche Fenster extrahiert. Dies ist eine Methode um mehr trainigsdaten zu generieren und das CNN darauf zu trainieren die Objekte nicht nur zentriert zu erkennen.
\end{itemize}
\vspace{2em}
\noindent\textit{load\_data\_by\_labels()} Parameter:
\begin{itemize}[leftmargin=*,labelindent=3cm, labelsep=1cm]
		\item[\textit{cam}] Int: 1 oder 2 von welcher der Kameras die Bilder geladen werden sollen. Wird dieser Parameter nicht verwendet werden Bilder beider Kameras verwendet.
		\item[\textit{no\_background}] Boolean: Es werden soviele, zufällig ausgewählte Hintergrundausschnitte aus einem Referenzbild geladen wie die negativ Klasse enthält, wenn True.
		\item[\textit{rotate\_negatives}] Boolean: rotiert alle Fenster der negativ Klasse 3 mal um 90\degree und spiegelt sie, wenn True.
		\item[\textit{rotate\_positives}] Boolean: rotiert alle Fenster der positiv Klasse 3 mal um 90\degree und spiegelt sie, wenn True.
\end{itemize}

\noindent mit diesen Möglichkeiten können die Daten in zwei Zeilen Code individuell für jedes Training vorbereitet werden.\\
\\
Vor dem Training werden die Daten dann zusätzlich noch normalisiert, damit alle Werte zwischen 0 und 1 liegen und das Datenset wird gemischt, dass beim Training nicht ganze Batches dieselbe Klasse repräsentieren.

\subsection{Aufbau des CNN}

Das \gls{CNN} wurde nach dem Vorbild von \parencite{cnnArchitecture} aufgebaut. Da diese Arbeit sich auf Bilder mit niedriger Auflösung spezialisiert. Das Netzwerk ist wie in Abbildung \ref{fig:cnnArchitecture} zu sehen. Es beginnt mit einem Convolutional Layer mit 64 5x5 Filter danach wird ein Maxpooling eingesetzt. Maxpooling funktioniert so, dass 

\begin{figure}[H]
	\centering
	\includegraphics[width=.5\linewidth]{MaxpoolSample}
	\caption{Visualisierung Maxpooling \parencite{MaxpoolImg2018} }
	\label{fig:maxpoolSample}
\end{figure}

\begin{figure}[H]
	\centering
	\includegraphics[width=.7\linewidth]{modelSummary}
	\caption{Architektur des CNN}
	\label{fig:cnnArchitecture}
\end{figure}





\chapter{Evaluation und Validation}
\label{ch:Eval}

Um das System zu validieren und die Grenzen der Infrarotbildverarbeitung in diesem Gebiet zu ermitteln, wurden einige Experimente durchgeführt. Diese Experimente zeigen, was das System kann und wo die Grenzen des Systems und auch der Infrarottechnik liegen.\\
\\
\section{Erklärungen}
\label{sec:explanation}
In diesem Kapitel werden Test statistisch ausgewertet und analysiert. Dazu werden hier die wichtigsten Begriffe und Konzepte erklärt.

\subsection{Farbcodierung}
Zur Auswertung der Experimente werden Bilder mit Markierungen verwendet. Diese sind wie folgt Farbcodiert.\\

\begin{itemize}
	\item \textbf{Grün:} Korrekt gefundene Person (True Positive)
	\item \textbf{Gelb:} Nicht gefundene Person (False Negative)
	\item \textbf{Rot:} Nichtiger Treffer (False Positive)
\end{itemize}
\vspace{1em}

\subsection{Begriffserklärung}
\noindent In dem Kapitel werden die folgenden Fachbegriffe verwendet:
\begin{itemize}
	\item $True\: Positive:$   Person wurde korrekt Identifiziert
	\item $True\: Negative:$   Hintergrund oder fremde Wärmequelle wurde \textbf{nicht} als Person identifiziert. True Negative werden in diesem Kapitel nicht weiter verwendet, da dies der Standardfall ist und nicht spezifisch ausgewertet werden kann.
	\item $False\: Positive:$  Hintergrund oder fremde Wärmequelle wurde als Person identifiziert
	\item $False\: Negative:$  Person wurde nicht erkannt.
\end{itemize}
\vspace{1em}

\subsection{Statistische Metriken}
\noindent Zudem werden die folgenden Metriken verwendet. Diese sind leicht angepasst, da die Klasse True Negative fehlt:

\begin{itemize}
	\item \textbf{Precision:} Korrekte Treffer / Alle Treffer\[\dfrac{True\: Positive}{True\: Positive + False\: Positive}\]
	\item \textbf{Recall/Accuracy:} Korrekte Treffer / Anzahl Personen\[\dfrac{True\: Positive}{True\: Positive + False\: Negative}\]
	\item \textbf{F1-Score:} \[\dfrac{2*Precision*Recall}{Precision + Recall}\]
\end{itemize}


\section{Vergleich mit Anforderungen}
\label{sec:VergleichAnforderungen}

Der erforderte Stand der Technik wurde in Kapitel \ref{ch:StandDerTechnik} präsentiert. Die gefundenen Algorithmen wurden in Kapitel \ref{ch:ideasAndConcepts} erklärt und evaluiert.\\
Das erste Hauptziel war es, ein lauffähiges System zu entwickeln, welches die Position und Anzahl der Personen in einem Infrarotbild bestimmen kann. Dieses Ziel wurde mit einem Recall von 91.5\%  erreicht (siehe Tabelle \ref{tbl:stat}). Beim Betrachten der Werte in Tabelle \ref{tbl:results} und \ref{tbl:stat} ist zu beachten, dass dabei auch die Grenzen des Systems getestet wurden. So zum Beispiel in Kapitel \ref{sec:cloths} wo der Einfluss isolierender Kleidung analysiert wird.\\
Das zweite Hauptziel ist es, Schwierigkeiten bei der Lösung der Aufgabe zu erläutern und Lösungsmöglichkeiten dazu vorzuschlagen. Dies wird in diesem Kapitel und im Kapitel \ref{ch:Ausblick} erfüllt.

% Tabelle mit ergebnissen
{
	\renewcommand{\arraystretch}{1.3}
	
	\begin{table}[H]
		\scriptsize
		\centering
		\begin{tabularx}{.5\textwidth}{Xrr}\\
			\hline
			\multicolumn{3}{c}{\textbf{Ergebnisse der Algorithmen}}\\
			\hline
			\textbf{Kategorien} & \textbf{CNN} & \textbf{Threshold}\\
			\hline
			\textit{Anzahl verwendeter Aufnahmen} & 323 & 323\\
			\hline 
			\textit{Gesamtanzahl Personen in Bilder} & 718 & 718\\
			\hline
			\textit{True Positive} & 657 & 650\\
			\hline
			\textit{False Positive} & 120 & 142\\
			\hline
			\textit{False Negative} & 61 & 68\\
		\end{tabularx}
		\caption{Zusammenfassung der Ergebnisse beider Algorithmen}
		\label{tbl:results}
	\end{table}
	\begin{table}[H]
		\scriptsize
		\centering
		\begin{tabularx}{.5\textwidth}{Xrr}
			\hline
			\multicolumn{3}{c}{\textbf{Statistische Auswertung}}\\
			\hline
			\textbf{Kategorien} & \textbf{CNN} & \textbf{Threshold}\\
			\hline
			\textit{Recall} & 91.5\% & 90.5\%\\
			\hline  
			\textit{Precision} & 84.5\% & 82.1\%\\
			\hline
			\textit{F1-Score} & 87.9\% & 86.1\%\\
			\hline
		\end{tabularx}
		\caption{Statistische Auswertung der Performance beider Algorithmen}
		\label{tbl:stat}
	\end{table}
}

\section{Fremde Wärmequellen}
\label{sec:FremdeWärmequellen}

In diesem Versuch wurde der Einfluss von fremden Wärmequellen, wie zum Beispiel Laptops, Natels, Kaffees oder Radiatoren, evaluiert. Dabei wurde der Effekt von Störquellen mit und ohne Personen im Raum analysiert.

\subsection{Versuchsaufbau}

Um die Performance der Algorithmen spezifisch in Bezug auf fremde Wärmequellen zu testen, wurden verschiedene Testsituationen erzeugt. Zuerst wurden fünf Laptops, von 13'' bis 17'', auf den Sitzungstisch gestellt, um zu sehen, ob Laptops ohne Personen Treffer generieren. Die Geräte wurden auf den gesamten Kamerawinkel verteilt, vom Zentrum des Bildes bis an den Rand. Da die Algorithmen nicht durch die Anzahl Wärmequellen, sondern nur durch die Distanz zwischen den Wärmequellen beeinflusst werden, konnte der ganze Kamerawinkel in einem einzigen Versuch getestet werden.\\
Die Temperatur der Laptops wurde von der Infrarotkamera als ca. 25\degree C erfasst, somit etwas unter der Abstrahlungswärme einer Person. Die Hauttemperatur einer Person wird von den Infrarotkameras zwischen 27\degree C und 33\degree C gemessen, je nach Kleidung und Behaarung.\\
Danach nahmen fünf Personen bei den Laptops Platz,  ohne dass die Laptops verschoben wurde. Um zu testen, ob False Positives generiert werden, wenn jemand den Laptop bedient, wurden die Probanden angeleitet eine Haltung einzunehmen, als würden sie die Laptops benutzen.\\
Um auch wärmere Laptops überprüfen zu können, wurden Bilder während einer Besprechung aufgezeichnet. Dabei war ein Laptop, dessen Temperatur im Bereich der Hauttemperatur einer Person liegt, in der Mitte des Tisches, ein Anderer nahe bei einer Person platziert. Die Szene ist in Abbildung \ref{fig:exampleDeviceTest2} zu sehen.\\
\\
\begin{figure}[htb]
	\centering
	\begin{subfigure}{.45\linewidth}
		\centering
		\includegraphics[keepaspectratio, height=4cm]{groundDeviceTest2}
	\end{subfigure}
	\begin{subfigure}{.45\linewidth}
		\centering
		\includegraphics[keepaspectratio, height=4cm]{exampleDeviceTest2}
	\end{subfigure}
	\caption{Ansicht des Sitzungszimmers während der Aufnahme des zusätzlichen Laptop- und Personen-Testsets}
	\label{fig:exampleDeviceTest2}
\end{figure}

\noindent
Bereits bei der Erstellung der Modelle wurde ersichtlich, dass das Herausfiltern von Wärmequellen mit sehr kleiner Fläche für beide Algorithmen kein Problem darstellt. Aus diesem Grund wurden die Versuche mit kleineren Wärmequellen nur in kleinem Rahmen durchgeführt. Es wurde ein Test mit einem Mobiltelefon durchgeführt, in dem ein Natel an verschiedenen Postionen auf den Tisch gelegt wurde. Da Mobiltelefone meist in der Hosentasche getragen werden, weisen diese Temperaturen bis zu 34\degree C auf.\\
\\

\subsection{Resultate}

{
	\renewcommand{\arraystretch}{1.3}
	\begin{table}[H]
		\centering
		\scriptsize
		\begin{tabularx}{.9\textwidth}{Xrrrr}
			\hline
			\multicolumn{5}{c}{\textbf{\gls{CNN} Ergebnisse: Fremde Wärmequellen}}\\
			\hline
			\textbf{Metriken} & \textbf{Nur Laptops} & \textbf{Laptops mit Personen} & \textbf{Mobiltelefon} & \textbf{Radiator}\\
			\hline 
			\textbf{Gesamtanzahl Personen in Bilder} & 0 & 39 & 0 & 20\\
			\hline
			\textbf{Korrekt identifizierter Personen} & 0 & 39 & 0 & 20\\
			\hline
			\textbf{Falsche Treffer} & 1 & 0 & 0 & 27\\
			\hline
			\textbf{Nicht erkannte Personen} & 0 & 0 & 0 & 0\\
			\hline
			\textbf{Recall} & -- & 100\% & -- & 100\%\\
			\hline  
			\textbf{Precision} & -- & 100\% & -- & 42.6\%\\
			\hline
			\textbf{F1-Score} & -- & 100\% & -- & 59.7\%\\
			\hline
		\end{tabularx}
		\caption{Ergebnisse des \gls{CNN}'s der Experimente mit fremden Wärmequellen}
		\label{tbl:heatSourcesCNN}
	\end{table}
	\begin{table}[H]
		\centering
		\scriptsize
		\begin{tabularx}{.9\textwidth}{Xrrrr}
			\hline
			\multicolumn{5}{c}{\textbf{Threshold-Methode: Ergebnisse Fremde Wärmequellen}}\\
			\hline
			\textbf{Metriken} & \textbf{Nur Laptops} & \textbf{Laptops mit Personen} & \textbf{Mobiltelefon} & \textbf{Radiator}\\
			\hline 
			\textbf{Gesamtanzahl Personen in Bilder} & 0 & 39 & 0 & 20\\
			\hline
			\textbf{Korrekt identifizierter Personen} & 0 & 33 & 0 & 20\\
			\hline
			\textbf{Falsche Treffer} & 1 & 014 & 16 & 87\\
			\hline
			\textbf{Nicht erkannte Personen} & 0 & 0 & 0 & 0\\
			\hline
			\textbf{Recall} & -- & 84.6\% & -- & 100\%\\
			\hline  
			\textbf{Precision} & -- & 70.2\% & -- & 18.7\%\\
			\hline
			\textbf{F1-Score} & -- & 76.7\% & -- & 31.5\%\\
			\hline
		\end{tabularx}
		\caption{Ergebnisse der Threshold-Methode der Experimente mit fremden Wärmequellen}
		\label{tbl:heatSourcesThresh}
	\end{table}
}

\subsection{Evaluation}

Die Threshold-Methode kann, wie erwartet, nicht gut mit anderen Wärmequellen umgehen. Dies, weil die Threshold-Methode nur durch Temperatur und Mindestgrösse eines Objekts entscheiden kann, ob es sich um eine Person handelt oder nicht. Deshalb hatte dieser auf den Testbildern mit Personen und Laptops, nur eine Precision Score von 70\% erreicht und bei dem Versuch mit Radiator auf hoher Stufe sogar nur eine von 18.7\% (Siehe Abbildungen  \ref{fig:ThreshPersonLaptop} und \ref{fig:thresholdRadiator}).\\
\\
Das \gls{CNN} kann gut mit anderen Wärmequellen umgehen, solange diese in ähnlicher Form antrainiert wurden (siehe Abbildung \ref{fig:cnnPersonLaptop}). Ist dies jedoch nicht der Fall und die Wärmequelle ist dem \gls{CNN} nicht in ähnlicher Form bekannt, tendiert es dazu, diese, wenn sie etwas grösser sind, als Personen zu identifizieren, was in Abbildung \ref{fig:cnnRadiator} zu sehen ist. Eine Lösung zu diesem Problem wird in Kapitel \ref{ch:Ausblick} diskutiert. 

\vspace{.5em}
\begin{figure}[htb]
	\centering
	\begin{subfigure}{.45\linewidth}
		\centering
		\includegraphics[keepaspectratio,height=4cm]{ThreshPersonLaptop}
		\caption{Ergebnis der Threshold-Methode des Experiments mit Laptop und Personen}
		\label{fig:ThreshPersonLaptop}
	\end{subfigure}\hfill%
	\begin{subfigure}{.45\linewidth}
		\centering
		\includegraphics[keepaspectratio,height=4cm]{threshRadiator}
		\caption{Ergebnis der Threshold-Methode des Radiatorexperiments}
		\label{fig:thresholdRadiator}
	\end{subfigure}\hfill%
	\begin{subfigure}{.45\linewidth}
		\centering
		\includegraphics[keepaspectratio,height=4cm]{cnnPersonLaptop}
		\caption{Ergebnis des CNN's des Experiments mit Laptops und Personen}
		\label{fig:cnnPersonLaptop}
	\end{subfigure}\hfill%
	\begin{subfigure}{.45\linewidth}
		\centering
		\includegraphics[keepaspectratio, height=4cm]{cnnRadiator}
		\caption{Ergebnis des CNN's des Radiatorexperiments}
		\label{fig:cnnRadiator}
	\end{subfigure}\hfill%
	\begin{subfigure}{.5\linewidth}
		\centering
		\includegraphics[keepaspectratio,height=3cm]{groundDeviceTest2}
		\caption{Ground-Truth des Experiments mit Laptop und Person}
		\label{fig:groundPersonLaptop}
	\end{subfigure}\hfill%
	\begin{subfigure}{.5\linewidth}
		\centering
		\includegraphics[keepaspectratio,height=3cm]{groundRadiator}
		\caption{Ground-Truth des Radiatorexperiments}
		\label{fig:groundRadiator}
	\end{subfigure}\hfill%
	\caption{Experimente mit fremden Wärmequellen}
	\label{fig:HeatSources}
\end{figure}
\vspace{.5em}



\section{Distanz}
\label{sec:distanz}

Die Distanz zwischen zwei Personen ist ein wichtiger Faktor, der auf beide Algorithmen Einfluss hat. Da bei dem \gls{CNN} mit einer Sliding-Window Methode und Clustern der Treffer gearbeitet wird, kann das System bei Personen, welche zu wenig Abstand zueinander haben, die Treffer nicht mehr auseinanderhalten. Bei der Threshold-Methode wird die Silhouette der Personen leicht vergrössert, wodurch mehrere Personen bei geringem Abstand als einzelner Treffer gewertet werden können.

\subsection{Versuchsaufbau}

Es wurden vier Personen in zwei Paaren so platziert, dass sich das eine Paar in einem idealen Winkel zur Infrarotkamera aufhält und das andere Paar in einem möglichst schwierigen Winkel. Dies ist in Abbildung \ref{fig:cnnDistance10} gut zu sehen. Zwei Personen sind deutlich voneinander getrennt, die anderen zwei überlappen sich aufgrund des Kamerawinkels. Diese Versuchsanordnung wurde gewählt, um in einem Test beide Extreme überprüfen zu können. 
Beim Test wurde schrittweise der Abstand zwischen den Testpersonen erhöht, um festzustellen, ab welcher Distanz die Algorithmen erfolgreich die Personen erkennen. Der Abstand wird in 10cm Schritten erhöht, weil 10cm auf Tischhöhe ca. einem Pixel auf dem Bild entspricht.

\subsection{Resultate}

{
	\renewcommand{\arraystretch}{1.3}
	\begin{table}[H]
		\centering
		\scriptsize
		\begin{tabularx}{.9\textwidth}{Xrrrrrr}
			\hline
			\multicolumn{7}{c}{\textbf{\gls{CNN} Ergebnisse Distanz Experiment}}\\
			\hline
			\textbf{Metriken} & \textbf{10cm} & \textbf{20cm} & \textbf{30cm} & \textbf{40cm} & \textbf{50cm} & \textbf{60cm}\\
			\hline 
			\textbf{Gesamtanzahl Personen in Bilder} & 29 & 18 & 54 & 18 & 51 & 28\\
			\hline
			\textbf{Korrekt identifizierter Personen} & 29 & 18 & 54 & 18 & 51 & 28\\
			\hline
			\textbf{Falsche Treffer} & 0 & 0 & 0 & 0 & 0 & 0\\
			\hline
			\textbf{Nicht erkannte Personen} & 0 & 0 & 0 & 0 & 0 & 0\\
			\hline
			\textbf{Recall} & 100\% & 100\% & 100\% & 100\% & 100\% & 100\%\\
			\hline  
			\textbf{Precision} & 100\% & 100\% & 100\% & 100\% & 100\% & 100\%\\
			\hline
			\textbf{F1-Score} & 100\% & 100\% & 100\% & 100\% & 100\% & 100\%\\
			\hline
		\end{tabularx}
		\caption{Ergebnisse des \gls{CNN}'s der Distanzexperimente}
		\label{tbl:distanceCNN}
	\end{table}
	\begin{table}[H]
		\centering
		\scriptsize
		\begin{tabularx}{.9\textwidth}{Xrrrrrr}
			\hline
			\multicolumn{7}{c}{\textbf{Threshold-Methode Ergebnisse Distanz Experiment}}\\
			\hline
			\textbf{Metriken} & \textbf{10cm} & \textbf{20cm} & \textbf{30cm} & \textbf{40cm} & \textbf{50cm} & \textbf{60cm}\\
			\hline 
			\textbf{Gesamtanzahl Personen in Bilder} & 29 & 18 & 54 & 18 & 51 & 28\\
			\hline
			\textbf{Korrekt identifizierter Personen} & 22 & 15 & 54 & 18 & 51 & 28\\
			\hline
			\textbf{Falsche Treffer} & 4 & 1 & 3 & 1 & 3 & 1\\
			\hline
			\textbf{Nicht erkannte Personen} & 7 & 3 & 0 & 0 & 0 & 0\\
			\hline
			\textbf{Recall} & 75.9\% & 83.3\% & 100\% & 100\% & 100\% & 100\%\\
			\hline  
			\textbf{Precision} & 84.6\% & 93.8\% & 94.7\% & 94.7\% & 94.4\% & 100\%\\
			\hline
			\textbf{F1-Score} & 80.0\% & 88.2\% & 97.2\% & 97.2\% & 97.1\% & 100\%\\
			\hline
		\end{tabularx}
		\caption{Ergebnisse der Threshold-Methode der Distanzexperimente}
		\label{tbl:distanceThresh}
	\end{table}
}

\subsection{Evaluation}


Das \gls{CNN} kann bereits ab einem Abstand von 10cm zuverlässig Personen auseinanderhalten.\\
Die Threshold-Methode kann Personen bereits ab einem Abstand von 10cm auseinanderhalten, sofern die Personen so ausgerichtet sind, dass die optische Verzerrung keinen Einfluss auf den Zwischenraum hat. Wie in der Abbildung \ref{fig:thresholdDistance10} in der linken Bildhälfte zu sehen ist, können die Personen, die auf die Kamera ausgerichtet sind, unterschieden werden. Sind die Person jedoch so platziert, wie auf der rechten Bildhälfte zu sehen ist, sind diese von der Kamera aus gesehen, hintereinander. Bei dieser geringen Distanz kann der Algorithmus deshalb nicht mehr bestimmen, ob es sich um eine oder mehrere Personen handelt.\\
Befinden sich die Personen in der schlechtestmöglichen Position, benötigt die Threshold-Methode einen Mindestabstand von 30cm, um die Personen fehlerfrei unterscheiden zu können (siehe Abbildung \ref{fig:thresholdDistance30}).

\begin{figure}[H]
	\begin{subfigure}{.45\linewidth}
		\centering
		\includegraphics[keepaspectratio,height=4cm]{CNNDistance10}
		\caption{\gls{CNN} Ergebnis des Distanztest mit 10cm Abstand}
		\label{fig:cnnDistance10}
	\end{subfigure}\hfill%
	\begin{subfigure}{.45\linewidth}
		\centering
		\includegraphics[keepaspectratio,height=4cm]{threshDistance10}
		\caption{Threshold Ergebnis des Distanztest mit 10cm Abstand}
		\label{fig:thresholdDistance10}
	\end{subfigure}\hfill
	\begin{subfigure}{\linewidth}
		\centering
		\includegraphics[keepaspectratio,height=3cm]{GroundDistance10}
		\caption{Ground-Truth Distanzexperiment mit 10cm Abstand}
		\label{fig:groundDistance10}
	\end{subfigure}
	\caption{Distanzexperiment 10cm}
	\label{fig:Distance10}
\end{figure}

\begin{figure}[H]
	\begin{subfigure}{.45\linewidth}
		\centering
		\includegraphics[keepaspectratio,height=4cm]{CNNDistance30}
		\caption{\gls{CNN} Ergebnis des Distanztest mit 30cm Abstand}
		\label{fig:cnnDistance30}
	\end{subfigure}\hfill%
	\begin{subfigure}{.45\linewidth}
		\centering
		\includegraphics[keepaspectratio,height=4cm]{threshDistance30}
		\caption{Threshold Ergebnis des Distanztest mit 30cm Abstand}
		\label{fig:thresholdDistance30}
	\end{subfigure}\hfill%
	\begin{subfigure}{\linewidth}
		\centering
		\includegraphics[keepaspectratio,height=3cm]{GroundDistance30}
		\caption{Ground-Truth}
		\label{fig:groundDistance30}
	\end{subfigure}
	\caption{Distanzexperiment 30cm}
	\label{fig:Distance30}
\end{figure}

\section{Grösse des Objekts auf dem Bild}
\label{sec:objectSize}
Eine Spezialität des Sitzungszimmers, das zur Verfügung gestellt wurde, ist, dass die Höhe der Decke verändert werden kann. Da die Infrarotkameras an der Decke montiert sind, wurde dies als Möglichkeit genutzt, um zu testen, wie die Algorithmen auf unterschiedliche Grössen der Objekte im Bild reagieren.

\subsection{Versuchsaufbau}

Die verstellbare Decke des Sitzungszimmers wurde auf die vom Mobiliar zugelassene Minimalhöhe, ca. 255cm, heruntergefahren. Danach wurden Infrarotbilder von Testpersonen an verschiedenen Positionen aufgenommen (siehe Abbildung \ref{fig:ground255cm}). Anschliessend wurde die Decke maximal hochgefahren, etwa 400cm, und noch einmal Bilder aufgezeichnet.

\begin{figure}[H]
	\centering
	\includegraphics[height=4cm]{ground255cm}
	\caption{Sitzungszimmer mit Decke auf 255cm}
	\label{fig:ground255cm}
\end{figure}

\subsection{Resultate}

{
	\renewcommand{\arraystretch}{1.3}
	\begin{table}[H]
		\scriptsize
		\centering
		\begin{tabularx}{.6\textwidth}{Xrr}
			\hline
			\multicolumn{3}{c}{\textbf{CNN Ergebnisse des Objektgrössenexperiments}}\\
			\hline
			\textbf{Metriken} & \textbf{255cm} & \textbf{400cm}\\
			\hline
			\textbf{Gesamtanzahl Personen in Bilder} & 78 & 44 \\
			\hline
			\textbf{Korrekt identifizierter Personen} & 77 & 42\\
			\hline
			\textbf{Falsche Treffer} & 7 & 1\\
			\hline
			\textbf{Nicht erkannte Personen} & 1 & 2\\
			\hline
			\textbf{Recall} & 98.7\% & 95.5\%\\
			\hline  
			\textbf{Precision} & 91.7\% & 97.7\%\\
			\hline
			\textbf{F1-Score} & 95.1\% & 96.6\%\\
			\hline
		\end{tabularx}
		\caption{CNN Ergebnisse des Distanztests}
		\label{tbl:objectSizeCNN}
	\end{table}
	\begin{table}[H]
		\scriptsize
		\centering
		\begin{tabularx}{.6\textwidth}{Xrr}
			\hline
			\multicolumn{3}{c}{\textbf{Threshold-Methode Ergebnisse des Objektgrössenexperiments}}\\
			\hline
			\textbf{Metriken} & \textbf{255cm} & \textbf{400cm}\\
			\hline
			\textbf{Gesamtanzahl Personen in Bilder} & 78 & 44 \\
			\hline
			\textbf{Korrekt identifizierter Personen} & 77 & 44\\
			\hline
			\textbf{Falsche Treffer} & 8 & 1\\
			\hline
			\textbf{Nicht erkannte Personen} & 1 & 0\\
			\hline
			\textbf{Recall} & 98.7\% & 100\%\\
			\hline  
			\textbf{Precision} & 90.6\% & 97.8\%\\
			\hline
			\textbf{F1-Score} & 94.5\% & 98.9\%\\
			\hline
		\end{tabularx}
		\caption{Threshold-Methode Ergebnisse des Distanztests}
		\label{tbl:objectSizeThresh}
	\end{table}
}

\subsection{Evaluation}
Beide Algorithmen können die Personen in beiden Situationen noch immer in über 95\% der Fälle identifizieren. Werden die Objekte jedoch zu gross, kann es bei beiden dazu führen, dass Personen doppelt gezählt werden. Da eine solch kleine Distanz aber nur schon durch das sehr kleine Sichtfeld für einen solchen Anwendungsfall nicht praktikabel ist, ist dies kein Problem, das weiter verfolgt werden müsste.\\
Werden die Objekte kleiner, stellt dies für beide Algorithmen kein messbares Problem dar, solange die Charakteristiken noch klar erkennbar sind und die Personen eine Fläche von mindestens 10x10 Pixel einnehmen. Ein Test um die Grenze in diesem Bereich zu testen konnte in den zur Verfügung stehenden Räumlichkeiten leider nicht durchgeführt werden.\\
Möchte man jedoch die reale Position der Personen aus dem Bild berechnen, müsste die Deckenhöhe zusätzlich berechnet oder gemessen werden.

\begin{figure}[H]
	\begin{subfigure}{.4\linewidth}
		\centering
		\includegraphics[keepaspectratio, height=4cm]{CNN255}
		\caption{Ergebnis des CNN mit Deckenhöhe 255cm}
		\label{fig:cnn255}
	\end{subfigure}\hfill%
	\begin{subfigure}{.4\linewidth}
		\centering
		\includegraphics[keepaspectratio, height=4cm]{thresh255}
		\caption{Ergebnis der Threshold-Methode mit Deckenhöhe 255cm}
		\label{fig:thresh255}
	\end{subfigure}\hfill%
	\begin{subfigure}{.4\linewidth}
		\centering
		\includegraphics[keepaspectratio, height=4cm]{CNN400}
		\caption{Ergebnis des CNN mit Deckenhöhe 400cm}
		\label{fig:cnn400}
	\end{subfigure}\hfill%
	\begin{subfigure}{.4\linewidth}
		\centering
		\includegraphics[keepaspectratio, height=4cm]{thresh400}
		\caption{Ergebnis der Threshold-Methode mit Deckenhöhe 400cm}
		\label{fig:thresh400}
	\end{subfigure}\hfill%
	\begin{subfigure}{.45\linewidth}
		\centering
		\includegraphics[keepaspectratio, height=4cm]{ground255cm}
		\caption{Ground Truth Deckenhöhe 255cm}
		\label{fig:ground255}
	\end{subfigure}\hfill%
	\begin{subfigure}{.45\linewidth}
		\centering
		\includegraphics[keepaspectratio, height=4cm]{ground400cm}
		\caption{Ground Truth Deckenhöhe 400cm}
		\label{fig:ground400}
	\end{subfigure}
\end{figure}


\section{Kleider}
\label{sec:cloths}

Infrarotbilder zeigen die Wärmeabstrahlung der verschiedenen Objekte im Bild. Wird eine Wärmequelle isoliert, verschwindet sie auf dem Infrarotbild. Genau das passiert, wenn stark isolierende Kleider getragen werden. Da in dieser Arbeit die Vogelperspektive analysiert wird, haben vor allem Kopfbedeckungen, Schals und Jacken einen grossen Einfluss. 

\subsection{Versuchsaufbau}

Eine Testperson hielt sich an den in den Abbildungen \ref{fig:clothPositions} zu sehenden Positionen im Raum auf und trug verschiedene Kleidungsstücke. Dabei wurden eine Mütze, ein Schal und eine Jacke einzeln und alle kombiniert von der Testperson getragen.

\begin{figure}[H]
	\centering
	\begin{subfigure}{.45\linewidth}
		\centering
		\includegraphics[keepaspectratio, height=3cm]{clothPos1}
		\caption{Position 1}
	\end{subfigure}
	\begin{subfigure}{.45\linewidth}
		\centering
		\includegraphics[keepaspectratio, height=3cm]{clothPos2}
		\caption{Position 2}
	\end{subfigure}
	\begin{subfigure}{.45\linewidth}
		\centering
		\includegraphics[keepaspectratio, height=3cm]{clothPos3}
		\caption{Position 3}
	\end{subfigure}
	\caption{Positionen der Testperson im Kleidungsexperiment}
	\label{fig:clothPositions}
\end{figure}

\subsection{Resultate}

{
	\renewcommand{\arraystretch}{1.3}
	\begin{table}[H]
		\centering
		\scriptsize
		\begin{tabularx}{.9\textwidth}{Xrrrr}
			\hline
			\multicolumn{5}{c}{\textbf{CNN Ergebnisse Kleidungsexperiment}}\\
			\hline
			\textbf{Metriken} & \textbf{Kappe} & \textbf{Jacke} & \textbf{Schal} & \textbf{Alle drei Kleidungsstücke}\\
			\hline 
			\textbf{Gesamtanzahl Personen in Bilder} & 23 & 18 & 21 & 3\\
			\hline
			\textbf{Korrekt identifizierter Personen} & 19 & 0 & 15 & 0\\
			\hline
			\textbf{Falsche Treffer} & 0 & 0 & 0 & 0\\
			\hline
			\textbf{Nicht erkannte Personen} & 4 & 18 & 6 & 3\\
			\hline
			\textbf{Recall} & 82.6\% & -- & 71.4\% & -- \\
			\hline  
			\textbf{Precision} & 100\% & -- & 100\% & -- \\
			\hline
			\textbf{F1-Score} & 90.5\% & -- & 83.3\% & -- \\
			\hline
		\end{tabularx}
		\caption{Ergebnisse des CNN des Kleidungsexperiments}
		\label{tbl:clothCNN}
	\end{table}
	\begin{table}[H]
		\centering
		\scriptsize
		\begin{tabularx}{.9\textwidth}{Xrrrr}
			\hline
			\multicolumn{5}{c}{\textbf{CNN Ergebnisse Kleidungsexperiment}}\\
			\hline
			\textbf{Metriken} & \textbf{Kappe} & \textbf{Jacke} & \textbf{Schal} & \textbf{Alle drei Kleidungsstücke}\\
			\hline 
			\textbf{Gesamtanzahl Personen in Bilder} & 23 & 18 & 21 & 3\\
			\hline
			\textbf{Korrekt identifizierter Personen} & 23 & 5 & 16 & 0\\
			\hline
			\textbf{Falsche Treffer} & 12 & 0 & 1 & 3\\
			\hline
			\textbf{Nicht erkannte Personen} & 0 & 13 & 5 & 0\\
			\hline
			\textbf{Recall} & 100\% & 27.8\% & 76.2\% & -- \\
			\hline  
			\textbf{Precision} & 65.7\% & 100\% & 94.1\% & -- \\
			\hline
			\textbf{F1-Score} & 79.3\% & 43.5\% & 84.2\% & -- \\
			\hline
		\end{tabularx}
		\caption{Ergebnisse der Threshold-Methode des Kleidungsexperiments}
		\label{tbl:clothThresh}
	\end{table}
}



\subsection{Evaluation}
Kleider beeinflussen die Performance der Algorithmen massiv. Trägt eine Person z.B. eine Mütze, wird sie vom \gls{CNN} noch zu 83\% erkannt. Bei der Threshold-Methode sind es zwar 100\%, es führt aber dazu, dass eine Person zwei Treffer erzeugt (siehe Abbildung \ref{fig:thresholdClothHat}).\\
\\
Wenn ein Schal getragen wird, sinkt der Recall auf 71\% beim \gls{CNN} und 76\% bei der Threshold-Methode. Betrachtet man Abbildung \ref{fig:scarfIR} sieht man deutlich, wie der Schal einen Teil der Wärme der Person verdeckt.\\
\\
Das Tragen einer Jacke macht es für die Algorithmen beinahe unmöglich die Personen zu erkennen. Die Person wird nur noch in wenigen Fällen erkannt. Da die Fläche der Abwärme einer Person so nur noch wenige Pixel beträgt, wird es schwierig diese von fremden Wärmequellen zu unterscheiden.\\

\begin{figure}[H]
	\centering
	\begin{subfigure}{.45\linewidth}
		\centering
		\includegraphics[keepaspectratio, height=4cm]{thresholdClothHat}
		\caption{Resultat der Threshold-Methode des Experiments mit Kopfbedeckung}
		\label{fig:thresholdClothHat}
	\end{subfigure}\hfill%
	\begin{subfigure}{.45\linewidth}
		\centering
		\includegraphics[keepaspectratio, height=4cm]{scarfIR}
		\caption{Infrarotbild einer Person die einen Schal trägt}
		\label{fig:scarfIR}
	\end{subfigure}\hfill%
	\caption{Infrarotbilder aus dem Kleidungsexperiment}
\end{figure}

\noindent
Die Tests mit allen Kleidungsstücken zeigen schon bei der manuellen, visuellen Analyse der Abbildung \ref{fig:rawClothAll}, dass wenn Jacke, Mütze und Schal zusammen getragen werden, dass die Person nicht mehr erkennbar ist.

\begin{figure}[H]
	\begin{subfigure}{.45\linewidth}
		\centering
		\includegraphics[keepaspectratio, height=4cm]{rawClothAll}
		\caption{Infrarotbild des Experiments mit Jacke, Mütze und Schal}
		\label{fig:rawClothAll}
	\end{subfigure}\hfill%
	\begin{subfigure}{.45\linewidth}
		\centering
		\includegraphics[keepaspectratio, height=4cm]{clothAll}
		\caption{Position der Person}
		\label{fig:AlgorithmsClothAll}
	\end{subfigure}\hfill%
	\begin{subfigure}{\linewidth}
		\centering
		\includegraphics[keepaspectratio, width=.5\linewidth]{GroundClothAll}
		\caption{Ground-Truth des Experiments mit allen Kleidern}
		\label{fig:groundTruthClothAll}
	\end{subfigure}
	\caption{Experimente mit Jacke, Mütze und Schal}
	\label{fig:AllCloth}
\end{figure}



\chapter{Ausblick}
\label{ch:Ausblick}

Zum Abschluss wird in diesem Kapitel über die Arbeit reflektiert und Diskutiert wie ein solches Projekt weitergeführt werden könnte und welche Verbesserungs- oder Erweiterungsmöglichkeiten es gibt.

\section{Projekt Fazit}

\newpage

\pagenumbering{Roman}

\appendix

\printglossary

\listoffigures

\listoftables

\pagebreak

\printbibliography
\end{document}
