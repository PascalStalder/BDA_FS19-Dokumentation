\chapter{Stand der Technik}
\label{ch:StandDerTechnik}

Die Bildverarbeitung im Bereich von Infrarot- und Temperaturbildern baut auf den erforschten Methodiken der klassischen Bildverarbeitung auf. Dies zeigt auch der Bericht einer Forschungsgruppe der ETH-Zürich\parencite{gomez2018thermal} . Diese stützt sich bei der Erarbeitung ihres Algorithmus zur Infrarot-Bildverarbeitung auf «A Convolutional Neural Network Cascade for Face Detection»\parencite{li2015convolutional} . Aus diesem Grund befindet sich im folgenden Stand der Forschung sowohl Algorithmen der klassischen Bildverarbeitung als auch solche die aus dem Feld der Verarbeitung von thermischen Aufnahmen\\
\\
Die momentan gängigsten Algorithmen sind \gls{CNN} \parencite{li2015convolutional} und verschiedene Filtermethoden wie zum Beispiel Hough-Transformationen\parencite{ye2015new}. Auch eine Kombination von \gls{CNN} und vorangehenden Filter sieht man oft. Zudem ist es gängige Praxis ein zu bearbeitendes Bild nicht als Ganzes zu verwenden, sondern jeweils nur einen kleinen Teil des Bildes zu verarbeiten. Dabei wird die Grösse dieses Fensters so gewählt, dass das grösstmögliche gesuchte Objekt vollständig abgebildet werden kann. Das Fenster wird in kleinen Schritten über das Bild bewegt um jeden dieser Bereiche zu analysieren. Dies ermöglicht eine sehr grobe Vorsortierung, bei der alle Teile die z.B. nur uninteressanten Hintergrund enthalten direkt verworfen werden können. Dabei wird gleichzeitig die Performance gesteigert und die Analyse der Interessanten Teile vereinfacht, da diese nun konzentrierter sind und nicht von der Position im Bild abhängig.\\
\\
\gls{CNN}’s trainieren vereinfacht ausgedrückt einen Vielschichtigen Filter, der die gesuchten Features bestmöglich repräsentiert. Da dieses Produkt als Bild für Menschen jedoch keinen Sinn ergeben würde, wird es direkt mit einer Aktivierungsfunktion in eine Wahrscheinlichkeit oder eine Klassifizierung umgewandelt.\\
Die Kaskadierung von \gls{CNN}’s ist eine Methode bei schrittweise komplexere und rechenintensivere \gls{CNN}’s nacheinander verwendet werden. Dies hat den Vorteil, dass Bildausschnitte die nur Hintergrund enthalten sehr schnell verworfen werden können. Um so wahrscheinlicher es ist einen Treffer zu haben um so bessere \gls{CNN}’s werden verwendet und so trotzdem eine sehr hohe Genauigkeit gewahrt.\\
\\
Um mittels Hough-Transformationen Formen zu erkennen wird zuerst mit einem Kantenerkennungsalgorithmus, wie z.B. der Canny-Algorithmus, alle Kanten in einem Binärbild repräsentiert. Im nächsten Schritt werden dann mittels Hough-Transformationen Formen wie Linien oder Kreise in diesem Binärbild identifiziert. Aus diesen Formen kann danach abgeleitet werden ob es sich um ein potenziell relevantes Objekt handelt.

Weiter gibt es auch noch Methoden wie \gls{k-NN} oder \gls{SVM}. Diese werden aber immer weniger verwendet, da sie in nahezu jedem Anwendungsfall von neuronalen Netzen überholt wurden. \gls{k-NN} werden oft als erster einfacher versuch oder als simpler Algorithmus für einfache Problemstellungen verwendet, sowie in Kombination mit anderen Methoden. \gls{SVM}'s waren lange die besten Kandidaten in Objekterkennungs-Wettbewerben wie zum Beispiel ImageNet \parencite{ILSVRC15}. Bis die nötige Rechenleistung verfügbar war um Tiefe \gls{CNN} zu trainieren. Seither gewannen in solchen Challanges fast ausschliesslich \gls{CNN}.

