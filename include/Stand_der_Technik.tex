\chapter{Stand der Technik}
\label{ch:StandDerTechnik}

Die Bildverarbeitung im Bereich von Infrarot- und Temperaturbildern baut auf den erforschten Methodiken der klassischen Bildverarbeitung auf. Dies zeigt auch der Bericht einer Forschungsgruppe der ETH-Zürich \parencite{gomez2018thermal}. Diese stützt sich bei der Erarbeitung ihres Algorithmus zur Infrarot-Bildverarbeitung auf «A Convolutional Neural Network Cascade for Face Detection» \parencite{li2015convolutional}. Aus diesem Grund befindet sich im folgenden Stand der Forschung sowohl Algorithmen der klassischen Bildverarbeitung als auch solche, die aus dem Feld der Verarbeitung von thermischen Aufnahmen stammen.\\
Es wurden einige Publikationen zu Objekterkennung im Infrarobereich gefunden, die auf die Verwendung von \gls{SVM} aufbauen \parencite{suard2006pedestrian, bertozzi2003pedestrian, zhang2007pedestrian}. Diese sind leider aufgrund ihres Alters problematisch. Alle diese Publikationen stammen von 2007 oder früher. Da aber um das Jahr 2011 bei der Mehrzahl von Wettbewerben \gls{SVM}'s von \gls{CNN}'s überholt wurden \parencite{Historyo5:online}, können diese nicht mehr vorbehaltlos als aktueller Stand der Technik in der Objekterkennung gewertet werden.

\section{Technologische Grundlagen}
\label{sec:technicalBase}
Die momentan gängigsten Algorithmen sind \gls{CNN} \parencite{li2015convolutional} und verschiedene Filtermethoden, wie zum Beispiel Hough-Transformationen \parencite{ye2015new}. Auch eine Kombination von \gls{CNN} mit vorangehenden Filtern sieht man oft. Zudem ist es gängige Praxis ein zu bearbeitendes Bild nicht als Ganzes zu verwenden, sondern jeweils nur einen kleinen Teil des Bildes zu verarbeiten. Dabei wird die Grösse dieses Fensters so gewählt, dass das grösstmögliche gesuchte Objekt vollständig abgebildet werden kann. Das Fenster wird in kleinen Schritten über das Bild bewegt, um jeden dieser Bereiche zu analysieren. Dies ermöglicht eine sehr grobe Vorsortierung, bei der alle Teile, die zum Beispiel nur uninteressanten Hintergrund enthalten, direkt verworfen werden können. Dabei wird gleichzeitig die Performance gesteigert und die Analyse der interessanten Teile vereinfacht, da diese nun konzentrierter sind und nicht von der Position im Bild abhängen.\\
\\
\gls{CNN}’s trainieren eine Vielzahl von Filtern, welche die gesuchten Features bestmöglich repräsentiert. Da dieses Produkt als Bild für Menschen jedoch keinen Sinn ergeben würde, wird es direkt mit einer Aktivierungsfunktion in eine Wahrscheinlichkeit oder eine Klassifizierung umgewandelt.\\
Die Kaskadierung von \gls{CNN}’s ist eine Methode, bei welcher schrittweise komplexere und rechenintensivere \gls{CNN}’s nacheinander verwendet werden. Dies hat den Vorteil, dass Bildausschnitte, die nur Hintergrund enthalten, sehr schnell durch simplere \gls{CNN}'s verworfen werden können. Entscheidet das simple \gls{CNN}, dass es sich nicht um Hintergrund handelt, werden komplexere \gls{CNN}’s verwendet, um trotzdem eine sehr hohe Präzision zu erhalten.\\
\\
Um mittels Hough-Transformationen Formen zu erkennen, wird zuerst mit einem Kantendetektionsalgorithmus, wie z.B. dem Canny-Algorithmus, ein Binärbild erstellt, welches alle Kanten des Originalbildes repräsentiert. Im nächsten Schritt werden dann mittels Hough-Transformationen Formen, wie Linien oder Kreise, in diesem Binärbild identifiziert. Aus diesen Formen kann danach abgeleitet werden, ob es sich um ein potenziell relevantes Objekt handelt oder nicht.\\
\\
Weiter gibt es auch noch Methoden wie \gls{k-NN} oder \gls{SVM}. Diese werden aber immer weniger verwendet, da sie in nahezu jedem Anwendungsfall von neuronalen Netzen überholt wurden. \gls{k-NN}'s werden oft als erste simple Kontrollversuche oder für einfache Problemstellungen, sowie in Kombination mit anderen Methoden, verwendet. \gls{SVM}'s waren lange die besten Kandidaten in Objekterkennungs-Wettbewerben, wie zum Beispiel ImageNet \parencite{ILSVRC15}. Seit die nötige Rechenleistung verfügbar ist, um tiefe \gls{CNN}'s zu trainieren, gewannen in solchen Challenges fast ausschliesslich \gls{CNN}'s. Momentan werden \gls{SVM}'s meist zur Unterstützung anderer Methoden eingesetzt.\\
\\
Im Bereich von \gls{CNN} gibt es viele Variationen wie \gls{yolo} oder der \gls{R-CNN} \parencite{yoloRCnn}, die sich auf das Detektieren von mehreren Objekten in einem Bild spezialisieren. Von diesen gibt es wiederum eine Vielzahl von Ableitungen die Geschwindigkeit oder Genauigkeit optimieren. Diese sind zwar meist auf hochauflösende Farbbilder spezialisiert, können aber konzeptuell auch in solch einem Projekt angewendet werden.

