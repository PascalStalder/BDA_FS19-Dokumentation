\chapter{Einleitung}

In diesem digitalen Zeitalter werden Ansprüche an Komfort und Energieeffizienz immer grösser. Damit auch moderne Gebäude diesen Ansprüchen gerecht werden müsse sie sich ständig weiterentwickeln und intelligenter werden.\\
Das ideale Gebäude hat stets die perfekte Temperatur, Luftfeuchtigkeit und Luftqualität, erreicht dies mit minimalem Energieaufwand und ohne dass der Nutzer dafür einen Finger rührt. Um dies zu erreichen, benötigt die Gebäudesteuerung Informationen mit denen sie die optimalen Bedingungen errechnen kann. Damit der Nutzer keinen Aufwand hat, müssen diese Informationen vollautomatisch gesammelt und verarbeitet werden.\\
Ein zusätzlicher Nutzen dieser Daten ist die Erstellung von Auslastungsstatistiken. Damit kann beispielsweise in einer Firma die Ausnutzung von Räumlichkeiten ausgewertet werden und mithilfe dieser Informationen, die Infrastruktur optimiert werden.

\section{Ausgangslage}
\label{sec:Ausgangslage}

Gebäudesteuerungen sollen in Zukunft mehr auf die Belegungen und das Verhalten der Nutzer eingehen, anstatt einem fixen Schema zu folgen. Voraussetzung dafür ist, dass die Steuerung, die Belegung und das Verhalten der Nutzer kennt.\\
Dafür soll in dieser Diplomarbeit das Potential kostengünstiger thermischer Kamerasystemen, zur Erfassung des Nutzerverhaltens in Büroräumen evaluiert werden. Mittels State-of-the-Art Bildanalyse und Deep-Learning soll versucht werden, die Belegung und den Personenfluss eines Sitzungszimmers zu bestimmen.\\
Ein Sitzungsraum der Hochschule Luzern in Horw wurde dafür mit zwei Infrarotkameras ausgerüstet. Zusätzlich steht eine herkömmliche Kamera als Referenz zur Verfügung.


\section{Zielsetzung}
\label{sec:Zielsetzung}

Ziel dieses Projekts ist es ein Modell zu erstellen, welches akkurat die Anzahl Personen und deren Position, in einem Raum, mittels einem Infrarotsensor, bestimmen kann. Dazu wird in einem ersten Schritt der Stand der Technik evaluiert und eine Methode für das weitere Vorgehen festgelegt. Danach wird diese Methode implementiert und unter realen Bedingungen getestet. Dabei soll Hauptsächlich gezeigt werden, ob ein solches System praktikabel ist und welche Herausforderungen oder Probleme es mit sich bringt. Zudem soll evaluiert werden wie diese Probleme überwunden werden können.

\section{Projektanforderungen}
\label{sec:Requirements}

Es sollen folgende Punkte erfüllt werden

\begin{itemize}
	\item Der Stand der Technik im Bereich Objekterkennung auf Infrarotbildern soll ermittelt werden.
	\item Die gefunden Algorithmen sollen evaluiert werden, nach ihrer Tauglichkeit für diese Problemstellung.
	\item Es soll ein lauffähiges System entwickelt werden, welche Position und Anzahl der Personen, auf einem Infrarotbild, ausgibt.\\
	Dazu sollen ein bis zwei Algorithmen ausgewählt und Implementiert werden.
	\item Es soll aufgezeigt werden, welche Schwierigkeiten bei der Entwicklung eines solchen Systems beachtet werden müssen und wie diese überwunden werden können.

\end{itemize}



