\chapter{Ideen und Konzepte}

Gemäss der Problemstellung sollten passende Algorithmen gesucht und Analysiert werden. Dazu wurde in der Ersten Projektphase eine breite Recherche durchgeführt wobei in erster Linie verschiedene Lösungsansätze gesucht wurden.

\subsection{Algorithmensuche}

Um Diese Aufgabe zu lösen wird ein bewährter und passender Algorithmus benötigt. Dazu wurde in der Recherchephase ermittelt welche Arten von Algorithmen für solche Fälle verwendet werden. Die gefundenen Algorithmen, werden dann evaluiert und bewertet so, dass schlussendlich ein Algorithmus ausgewählt werden kann der bestmöglich zur Lösung der Aufgabe verwendet werden kann.

\subsection{Clustering von Wärmebereichen}

Es werden Cluster von Pixeln über einem bestimmten Temperatur-Threshold gebildet. diese werden einzeln analysiert und anhand von Temperaturverteilung innerhalb des Clusters und der Form des Clusters wird entschieden ob es sich um eine Person handelt. Sind die Cluster evaluiert und markiert als Person oder keine Person können die Zentren der Person-Cluster direkt als Treffer verwendet werden.


\subsection{K-Nearest Neighbors}

\gls{k-NN}'s ist ein simpler Algorithmus das Datenpunkte vergleicht und danach entscheidet welche die grösste Ähnlichkeit besitzen. dieser Algorithmus wird ähnlich dem \gls{CNN} mit Referenzdaten bestückt und entscheidet daraufhin zu welcher Klasse ein Bild die grösste Ähnlichkeit aufweist. 

\subsection{Hough-Transformationen}

Hough-Transormationen erkennen Formen, wie Kreise und Linien auf Bildern. Dazu muss das Bild zuerst mittels ein Kantendetektionsalgorithmus so bearbeitet werden, dass nur noch kanten zu sehen sind. Danach extrahiert man mittel Hough-Transformationen Kreise und Linien. Diese Methodik könnte in diesem Projekt zur Unterstützung andere Algorithmen verwendet werden.


\subsection{Thresholding}

Beim Thresholding wird das Infrarotbild mittels mehreren fixierten Werten in ein Binärbild umgewandelt in welchem dann nur noch weisse Flecken zu sehen sein sollten, die Personen repräsentieren. Dazu wird in diesem Fall eine Mindest- und Maximaltemperatur festgelegt. Alle Pixel die innerhalb dieses Bereichs liegen, werden weiss eingefärbt, alle anderen schwarz. Danach wird dieses Bild erodiert und dilatiert. Bei der Erosion wird mit einer kleine Matrix, ein sogenannter Kernel, über das Binärbild iteriert. Immer, wenn mindestens ein Feld des Kernels schwarz ist wird die gesamte Fläche des Kernels, auf dem resultierenden Bild schwarz eingefärbt. die Dilatation funktioniert nach dem gleichen Prinzip aber arbeitet dabei mit den weissen Pixel.

\subsection{Convolutional Neural Network}

Während der Recherche stellte sich heraus, dass das \gls{CNN} für diesen Bereich der Objekterkennung, eine sehr beliebte Variante ist. Die Mehrheit der Arbeiten, die zu diesem Thema gefunden wurden, verwendeten \gls{CNN} in irgend einer Form.\\
\gls{CNN} funktionieren wie in Kapitel \ref{sec:technicalBase} erklärt, indem sie Filter trainieren, welche aus dem ursprünglichen Bild die nötigen Informationen extrahieren, um zu bestimmen, zu welcher Klasse das Bild gehört. Um dieses Training zu ermöglichen werden sehr viele Trainingsdaten benötigt.

\section{Datensammlung}

Da einige der erwähnten Methoden einen relativ grossen Datensatz benötigen, um Trainiert zu werden, stellen sich die Fragen wie diese gesammelt und vor allem, wie diese gelabelt werden. Das Sammeln an sich kann durch ein simples Skript realisiert werden, indem die Infrarotbilder periodisch von den Infrarotkameras angefordert und abgespeichert werden. Zusätzlich wurden parallel dazu auch Bilder einer optischen Kamera im Raum persistiert.\\
Das Labeln der Infrarotbilder konnte auf zwei Arten umgesetzt werden. Man Labelt alle Bilder die verwendet werden sollen manuell mit Hilfe eines Labeling Tools. Oder man erstellt ein Komplexes Program, das mittels 'State of the Art' Bildverarbeitung aus den optischen Bilder die Personen erkennt und deren Position auf die Position im Infrarotbild umrechnet.\\
\\
Dieser zweite Ansatz wäre vollautomatisch, schnell und wiederverwendbar, jedoch wäre die Erstellung eines solchen Programms sehr aufwendig und es könnte auch nicht garantiert werden, dass alle Personen korrekt markiert wurden. Es müsste also trotzdem Manuell kontrolliert werden, ob das Labeln korrekt ablief. 

